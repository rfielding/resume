\documentclass[margin]{res}
\setlength{\textwidth}{5.1in}

\begin{document}
\begin{resume}
\section{Contact}
Rob Fielding \\
rob.fielding@gmail.com \\
http://github.com/rfielding - Public code repository (nothing work related) \\
http://rfieldin.appspot.com - Demonstration of my work with Wizdom - high performance mobile development 

\section{Summary}
I am currently working at firewall vendor Check Point Software,
working in the context of Intrusion Prevention and general Security.
My experience spans from J2EE and web development and thick client
C\# Windows, to low level C in Linux servers.
I have been using Java since HotJava Alpha came out, and have undergone
Sun Java2 certification.

\section{Education}
George Mason University - Computer Science, graduated 1996 \\
Focus on algorithms, 3d graphics, assembly language, C \\
One of the earliest Java users (HotJava) 

\section{Skills}
Used professionally on a daily basis: C/C++, Java, C\#, Python, Javascript, PHP, Objective-C, graphviz, VMware (some use of the APIs), git. \\
Also used: Go, Erlang, OCaml/Haskell, ExtJS, Cassandra/Thrift, Solr, Hadoop MapReduce, Django, Spring/Hibernate. 

\section{Employment}
\subsection{2007-Present: Check Point Software Technologies}
Check Point acquired NFR Security (roughly 20 employees) \\
Currently working on a (in kernel module) virtual machine language for use within the firewall,
and IP packet parsing in a specialized hardware accelerator \\
Managed Security Services Portal, which started out as a PHP/Postgres
site built by sales engineers and handed to us to scale up before
new customer load created a performance problem; a classic high throughput
streaming application (incoming firewall logs from all over the world).
This used a Postgres database a 2Tb working set (not including the replica),
where we had to solve a lot of big performance problems while not violating the SLAs.  It was migrated to use a J2EE server backend, and
much of the database schema reworked to be much more efficient. 
Worked on a C\#/WPF based IDE and debugger for our custom language (cpcode/ncode) \\
Sensornet work involved ExtJS and cgi, writing MapReduce
jobs to generate IP connectivity graphs \\
My group is the group that introduced git usage into the company,
so we are familiar with the difficult parts of using it on a daily basis
(dealing with the consequences of choosing merge in cases where there
should have been a rebase, interactive rebasing to create clean histories, etc),we were using Gerrit as our workflow for about a year as of last year. \\

IPS-1 Security Alerting Dashboard (continuation of Sentivist Server) \\
We had done research into a J2EE backend plus C\#/WPF infrastructure
to handle automatically exposing async client APIs to sync remote EJBs,
using Hibernate/Spring as the backend, and writing a C\# implementation
of JavaSerialization to support EJB's RMI usage.

\subsection{2011: Wizdom Music (Part Time)}
iPad/iPhone development of a real-time music instrument for professional use. \\
The app names were: Mugician, Geo Synthesizer, and Cantor. \\
Geo Synthesizer was not only useful and popular, but financially successful for our team of three people (A well known rock musician with his programmer, and myself). \\
This work involved solving a lot of performance problems in C code.
Battery life, memory consumption, SIMD parallelism, signal processing,
and hard real-time constraints dominated every aspect of this design
and development. \\
This experience involved a lot of running the business directly, and involving customers directly in design and maintenance. \\
The source code for all of this, except for Geo Synthesizer (which may now be owned by Roli Labs), is in github.

\subsection{2004-2007: NFR Security (aka Network Flight Recorder)}
Sentivist Server - a high throughput (MySQL tens of millions of alerts per day 40Gb working set, per sensor device) JBoss/J2EE backed application server,
deeply involved in every aspect of its implementation and maintenance \\
Sentivist Dashboard - a Java Swing client that connected to Sentivist Server,
deeply involved in every aspect of it. \\
We got a patent related to viewing streaming data: 11/586,689, one
of many useful innovations in the security alerting dashboard \\
Because we were dealing primarily with high throughput streaming logs,
our J2EE app was largely based on JMS and merging it with queries to
give us an API similar to what is seen in Esper today.  \\
We also had to make some rather drastic changes to Jboss4.x so that we could
have the client connect correctly to the server across NATs, tunnels, and port forwards; while completely eliminating cleartext connections between client and server.  For performance reasons, this involved changing things to use immutable data structures that resemble modern CRDTs, and using the largest batches that latency constraints would allow.  We had to rewrite the transaction system
to avoid deadlock states in the MySQL server and in objects running in
the JVM (ie: calls to remote EJBs needed all remote calls to declare every lock that could be taken so that they all get taken in a globally consistent ordering).   This meant that our "J2EE" application was rather unorthodox,
not using J2EE persistence, using POJO to database APIs that correctly handled asynchronous design and efficient serialization; making our app more resemble an Erlang application than a typical J2EE application.

\subsection{1999-2004: Logicon / Northrop Grumman / ComTek / DISA}
Worked on multi-machine application suites primarily providing APIs
wrapping up LDAP for DoD PKI applications.  This involved client-side
SSL authentication (software and smartcard tokens).  
At one point, installations got reduced from 30 days due to waiting periods
for certificate signers happening at random installation points, down to
a few days by building a tool to request all future required certificates in one step and write them to the formats that the applications will demand later on \\
Worked on SAML authentication based applications related to DoD PKI \\
Taught class on encryption and PKI (similar notation to Bruce Schneier's Applied Cryptography group, but more group theory oriented notation), authored a lot of the documentation related to demonstrating that we took required measures to secure our code \\
Authored a JavaServlet based UI toolkit (similar to Echo2 framework before it became JavaScript based) to deal with scenarios where JavaScript
was not allowed for security reasons, yet have a rich widget API which was used in our applications. \\
DoD Secret clearance

\subsection{1997-1999: Federal Reserve Board Of Governors}
Worked on banking surveillance applications.
Generally, work was done in Powerbuilder (and occasional bits of Win32 C API).
While I was there, I introduced the use of Java for these kinds of applications,
and the first use of Java in production was by way of Silverlight.
 
\section{Other}
Willing to travel.  (My current work has had quite a bit of international travel.  Usually to work on-site with remote co-workers.) \\
Though I am currently doing rather low level work, 
my interests have been centered around good high level tools
such as strongly typed languages, and using other measures
to have machine checkable confidence in code (ie: DBC, LanguageTheoreticSecurity, etc). \\
Relevant to medical device companies - I am a type 1 (from type 2) diabetic, and have written code to parse my own pump data before; so I have practical experience as an end user of diabetic related applications.
 
\end{resume}
\end{document}

